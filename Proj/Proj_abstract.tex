%Do not change 
\documentclass[12pt, oneside]{article}
\usepackage{amssymb,amsmath}
\usepackage[margin=1in]{geometry}
\usepackage{textpos}

% You may add the packages you need here



\begin{document}
% Do not modify 
\begin{textblock*}{3cm}(-1.7cm,-2.6cm)
\noindent {\scriptsize Staple here!} 
\end{textblock*}

%Do not modify
\begin{textblock*}{4cm}(-1.7cm,-2.3cm)
\noindent {\scriptsize TTIC 31050 Winter 2018} 
\end{textblock*}

%Do not modify other than putting your name where stated
\begin{textblock*}{8cm}(12.5cm,-1cm)
\noindent {Name: Graham Northrup} 
\end{textblock*}
%Do not modify other than putting your section (1 or 2) where stated
\begin{textblock*}{3cm}(12.5cm,-0.5cm)
\noindent {Section: 1} 
\end{textblock*}
%Do not modify other than typing your acknowledgement where stated
%\begin{textblock*}{13.5cm}(-1.7cm,-1.8cm)
%\noindent \textit{\footnotesize Acknowledgement: Your acknowledgement for collaboration and other sources goes here. } 
%\end{textblock*}

\vspace{1cm}

%Do not modify other than typing the homework number after #
\begin{center}
\textbf{\Large Phylogenetics Project Abstract}
\end{center}


%Rest should contain your solution for the homework. Feel free to improvise in ways that you believe make grading easier.
\subsection*{Abstract} 
Phylogenetic tree construction represents a complex problem for evolutionary biologists. We can imagine how things such as homoplasies, horizontal gene transfer, continuous characteristics, and even missing data can all impact the resulting phylogenetic tree. At each step there are multiple decisions to be made, all the way at the beginning of creating a rooted or unrooted tree, as well as using phenotypic traits or molecular and genotypic traits. In this paper we will seek to survey the current algorithms used as well as future and current directions of development being used. We will look at distance-matrix methods, maximum parsimony (fewest evolutionary events) methods, maximum likelihood, and Bayesian methods. In addition to selecting how a tree is constructed we have access to multiple measures of support for the tree, which we will also discuss briefly. Ultimately phylogenetic tree construction is far from a solved problem, but it is important to survey all available methods in order to select the best one for a particular use.

\end{document}